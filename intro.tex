\section{Introduction}

Hadoop~\cite{hadoop} is a framework that allows distributed processing of large data sets across clusters of computers with simple MapReduce programming models. It is designed to scale up from single servers to thousands of machines, while each offering local computation and storage. Nowadays, it has been deployed by lots of companies (e.g., Facebook~\cite{facebook}, Amazon~\cite{amazon}, Ebay~\cite{ebay}, Yahoo~\cite{yahoo}). There are also many other applications built on top of Hadoop(e.g., Hive~\cite{thusoo2009hive,hive}, Hbase~\cite{hbase}, Pig~\cite{pig}, ZooKeeper~\cite{zookeeper}).

Hadoop was targeted to use on private clusters at the beginning and thus was not built with security in mind. The focus of Hadoop at that time is improving its efficiency. However, with the popularity of cloud computing, there emerges the requirement of running Hadoop on shared multi-tenant machines with sensitive data. The previous Hadoop could not meet this demand. Therefore, a secure mode for Hadoop is proposed and implemented in current version of Hadoop. The secure mode~\cite{securemode} consists of user authentication to protect Hadoop from unauthorized access to HDFS~\cite{shvachko2010hadoop} data, data confidentiality to protect transferred data between Hadoop nodes and clients from being intercepted by attackers, and delegation tokens to protect Hadoop from unauthorized access to MapReduce tasks.
%While a lot of work , people are gradually paying attention to its security issues and then building security modules for Hadoop. However, there exists no structured evaluation of security vulnerabilities in Hadoop considering the security enabled module.

However, secure mode has a disadvantage. It does not provide security isolation for MapReduce jobs. Hadoop executes user-submitted code without inspection. Malicious use could submit jobs with code to compromise Hadoop nodes. Therefore, containerization technology is incorporated into Hadoop. The most recently adopted technology is Docker~\cite{docker}. Docker provides resource isolation for MapReduce jobs in Docker containers. Each container is an independent runtime environment with its own view of namespace. The compromised Docker container could neither affect other Docker containers nor the host system which runs the Docker services. Thus applying the Docker container into Hadoop through the Docker Container Executor(DCE)~\cite{dce}could limits the damage of malicious code.
%Docker~\cite{docker} combines an easy-to-use interface to Linux containers with easy-to-construct image files for those containers. In short, Docker launches very light weight virtual machines. Containers take advantage of the Linux kernel’s ability to create isolated environments that are often described as a "chroot on steroids”~\cite{schmidt2006high}. Containers and the underlying host share a kernel. However, each container is assigned its own mostly independent runtime environment as with the help of Control Groups (cgroups)~\cite{cgroups} and namespaces. Each container receives its own network stack and process space, as well as its instance of a file system. However, at the moment, Docker does not provide each container with its own user namespace, which means that the launched user is only root and it offers no user ID isolation.
There are two security problems with DCE. One big problem is that it is incompatible with Hadoop secure mode~\cite{dce}. The reason lies in user authentication enabled by Kerberos in Hadoop secure mode. All nodes and users running Hadoop services are registered as Kerberos principals so that  Hadoop services could read their permissions in authorization configuration files. Since the Docker container is created and terminated dynamically, it is very difficult for Hadoop secure mode to register the Docker container and define their permissions. The other problem is introduced by a general security problem for Docker. The entire stack of Docker technology runs as a single process and requires root access to the machine. To use the Docker container, the permission of running Hadoop has to escalate to root.
%The latest activity of Hadoop community is the involvement of Docker through Docker Container Executor(DCE)~\cite{dce}. DCE allows the YARN~\cite{yarn} NodeManager to launch YARN containers into Docker containers. These containers provide a custom software environment in which the user's code runs, isolated from the software environment of the NodeManager. Docker for YARN provides both consistency and isolation because all YARN containers will have the same software environment and there is no interference with whatever is installed on the physical machine.

In this paper, we investigate the implication of the Hadoop's secure mode incompatibility with Docker. Specifically, we want to answer the following three questions:
%examined the combination of Docker container with Hadoop from the viewpoint of Hadoop security. 
\begin{itemize}
\item {How does container work in Hadoop?}
\item {What are the security problems have been solved by YARN container?}
\item {What are the security problems have not been solved yet?}
\end{itemize}

We discuss the current security concerns (i.e., arbitrary code execution, malicious user impersonation, and existing compromised nodes), and evaluate the influence of the combination on security modules. We then explore the introduced risks by evaluating the requirements for using the Docker container. Finally, we discuss the attack surface trade-offs and propose designs that is used to solve the problems that we have mentioned above.

The rest of the paper proceeds as follows. We introduce Hadoop framework and Docker container in Section \ref{sec:overview}. In Section \ref{security}, we discuss the security vulnerabilities and also security modules of Hadoop. Section \ref{sec:eval} presents our experiments to evaluate Docker container in Hadoop. Section \ref{sec:design} discusses the trade-off and proposed designs. Section \ref{sec:relwork} presents the related work. Finally, the paper concludes in Section \ref{sec:conc}.
