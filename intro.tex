\section{Introduction}

Hadoop~\cite{shvachko2010hadoop} is a framework that allows for the distributed processing of large data sets across clusters of computers using simple programming models. It is designed to scale up from single servers to thousands of machines, while each offering local computation and storage. It has been deployed by lots of companies, such as Facebook, Amazon, Adobe, Ebay, etc. There are also many other softwares built on top of Hadoop(e.g., Hive~\cite{thusoo2009hive,hive}, Hbase~\cite{hbase}, Pig~\cite{pig}, ZooKeeper~\cite{zookeeper}).

At the beginning, Hadoop was built with security in mind. While a lot of work on Hadoop has focused on improving its efficiency, people are gradually paying attention to its security concerns and building security modules for Hadoop. However, there exists no structured evaluation of security vulnerabilities in Hadoop considering the security enabled module.

Docker~\cite{docker} combines an easy-to-use interface to Linux containers with easy-to-construct image files for those containers. In short, Docker launches very light weight virtual machines. Containers take advantage of the Linux kernel’s ability to create isolated environments that are often described as a "chroot on steroids”~\cite{schmidt2006high}. Containers and the underlying host share a kernel. However, each container is assigned its own mostly independent runtime environment as with the help of Control Groups (cgroups)~\cite{cgroups} and namespaces. Each container receives its own network stack and process space, as well as its instance of a file system. However, at the moment, Docker does not provide each container with its own user namespace, which means that the launched user is only root and it offers no user ID isolation.

The latest activity of Hadoop community is the involvement of Docker through Docker Container Executor(DCE)~\cite{dce}. DCE allows the YARN~\cite{yarn} NodeManager to launch YARN containers into Docker containers. These containers provide a custom software environment in which the user's code runs, isolated from the software environment of the NodeManager. Docker for YARN provides both consistency and isolation because all YARN containers will have the same software environment and there is no interference with whatever is installed on the physical machine.

In this paper, we examined the combination of Docker container with Hadoop from the viewpoint of Hadoop security. Specifically, we want to answer the following three questions:

\begin{itemize}
\item {How does container work in Hadoop?}
\item {What are the security problems have been solved by YARN container?}
\item {What are the security problems have not been solved yet?}
\end{itemize}

We went through the current security concerns, namely arbitrary code execution, malicious user impersonation, existing compromised nodes; evaluated the influence of the combination on security modules; and explored introduced risks by evaluating the requirements for using Docker container.

The rest of the paper proceeds as follows. We introduce Hadoop framework and Docker container in Section 2. In Section 3, we discuss the security vulnerabilities and also security modules of Hadoop. Section 4 presents our experiments to evaluate Docker container in Hadoop. Section 5 discusses related work. Finally, the paper concludes in Section 6.
