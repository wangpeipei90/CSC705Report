\section{Evaluation}
\label{sec:eval}

\iffalse
Evaluation (don't forget to interpret your data)
\fi

\subsection{Experiment platform}

The experiment platform is shown in the table.



\subsection{Hadoop setup with Docker configuration}

We suppose that Hadoop is already installed in Pseudo-Distributed mode, and Docker is also installed. The path of docker command is /usr/bin/command.

The first step is to download docker image for Hadoop docker container. The command is: sudo docker pull sequenceiq/hadoop-docker:2.6.0

Next, add the following properties to yarn-site.xml:



In order for Hadoop to use Docker container, we need to change the permission of docker command with “sudo chmod u+s /usr/bin/docker”.

Go to the directory of Hadoop installtion, and start Hadoop with “sbin/start-dfs.sh” and “sbin/start-yarn.sh”. 

The results of  “jps” command will present you the running processes: NameNode, SecondaryNameNode, DataNode, ResourceManager, NodeManager.

To test the usage of Docker container in Hadoop, we use one Hadoop example program teragen to generate 1000 bytes of data. The command is as follows:

\iffalse
bin/hadoop jar share/hadoop/mapreduce/hadoop-mapreduce-examples-2.6.0.jar teragen -Dmapreduce.map.env="yarn.nodemanager.docker-container-executor.image-name=sequenceiq/hadoop-docker:2.6.0" -Dyarn.app.mapreduce.am.env="yarn.nodemanager.docker-container-executor.image-name=sequenceiq/hadoop-docker:2.6.0" 1000 teragen_out_dir
\fi

After the job is launched to execute, “docker ps” command shows that there are 3 containers that are launched to run the Teragen program. The 3 containers terminate after the job is finished.

\subsection{Example attack}

Since docker is being able to run by any user, use “docker stop containerID” could terminate the containers which are running MapReduce jobs. In this way, the malicious user could introduce failures to jobs. The screenshot is shown as below.